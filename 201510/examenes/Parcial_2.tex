\documentclass[letterpaper,10pt,onecolumn]{article}
\usepackage[spanish]{babel}
\usepackage[latin1]{inputenc}
\usepackage{amsfonts}
\usepackage{amsthm}
\usepackage{amsmath}
\usepackage{mathrsfs}
\usepackage{empheq}
\usepackage{enumitem}
\usepackage[pdftex]{color,graphicx}
\usepackage{hyperref}
\usepackage{listings}
\usepackage{calligra}
\usepackage{algpseudocode} 
\DeclareMathAlphabet{\mathcalligra}{T1}{calligra}{m}{n}
\DeclareFontShape{T1}{calligra}{m}{n}{<->s*[2.2]callig15}{}
\newcommand{\scripty}[1]{\ensuremath{\mathcalligra{#1}}}
\lstloadlanguages{[5.2]Mathematica}
\setlength{\oddsidemargin}{0cm}
\setlength{\textwidth}{490pt}
\setlength{\textheight}{610pt}
\setlength{\topmargin}{-85pt}
\addtolength{\hoffset}{-0.3cm}
\addtolength{\textheight}{4cm}

\begin{document}
\thispagestyle{empty}
\begin{center}

\includegraphics[width=490pt]{figs/header.png}\\[0.5cm]

\textsc{\LARGE Parcial 2 - F\'isica I (FISI-1018) - Feb. 19, 2015}\\[0.5cm]

%\textsc{\Large{Profesor: Jaime Forero --- Fecha: Febrero 19, 2015}} \\[0.5cm]
\end{center}

\begin{enumerate}
\item (20 puntos) Dos bloques se encuentran unidos por una cuerda sin
  masa, ubicados sobre una cu\~na que se encuentra fija al
  piso como  muestra la Figura. Encuentre la magnitud de la
  aceleraci\'on de los bloques si 
  $m_1=1$kg, $m_2=2$kg, $\alpha=45^{\circ}$ y $\beta=45^{\circ}$.


\item (20 puntos) Una cuerda sin masa y de longitud $l$ tiene en su extremo una
  masa $m$. Con esto se arma un p\'endulo c\'onico como muestra la
  Figura. La m\'axima tensi\'on que puede soportar la cuerda es
  $2mg$. �Cu\'anto es la velocidad m\'axima que puede tener la masa
  antes de que se rompa la cuerda si $m=1$kg y $l=1$m?  

\item (20 puntos)Un adulto de masa $M$ sostiene a un ni\~no de masa $m$ con una
  cuerda a trav\'es de una polea como se muestra en la
  Figura. �Cu\'anto vale la normal que hace el piso sobre el adulto si
  $M=70$kg y $m=$20kg?

\item (20 puntos)Dos cuerdas de masa $m$ sostienen un bloque de masa
  $M$ como muestra la Figura �Cu\'anto vale la tensi\'on en la mitad
  de una de las dos cuerdas si $m=100$gramos y $M=1$kg? 

\item (20 puntos) Un bloque cuadrado de masa $m$ se encuentra sobre
  una cu\~na triangular de masa $M$ y pendiente $\theta$. La cu\~na se
  encuentra en contacto con el piso y sobre ella act\'ua una fuerza
  $F$. Entre el bloque y la cu\~na hay fricci\'on (coeficiente
  est\'atico $\mu$, con $\mu<\tan\phi$). Entre la cu\~na y el suelo no hay
  fricci\'on. �Cu\'anto vale la fuerza {\bf m\'axima} $F$ que puedo aplicar
  de tal manera que el bloque se mueva junto a la cu\~na sin
  deslizarse? 

\item (20 puntos) Tenemos la configuraci\'on de bloques mostrada en la
  Figura. Entre los bloques hay fricci\'on con coeficiente est\'atico
  $\mu$, pero entre los bloques y el piso no hay fricci\'on. �Cu\'al
  es la fuerza  {\bf m\'axima} $F$ que se puede hacer antes de que el
  bloque $m_1$ a la derecha empieze a deslizarse sobre el bloque
  $m_2$? 

\end{enumerate}

\begin{figure}[!h]
\begin{center}
\includegraphics[scale=0.25]{figs/parcial2_1.png}
\includegraphics[scale=0.25]{figs/parcial2_2.png}
\includegraphics[scale=0.25]{figs/parcial2_3.png}
\includegraphics[scale=0.25]{figs/parcial2_4.png}
\includegraphics[scale=0.24]{figs/parcial2_5.png}
\includegraphics[scale=0.24]{figs/parcial2_6.png}
\caption{Figuras para cada uno de los ejercicios.}
\end{center}
\end{figure}
{\small {\bf NOTA}: Todas las respuestas deben tener una justificaci\'on
f\'isica y matem\'atica adecuada. Tome $g=10$ m/s$^{2}$. 100 puntos
corresponden a una calificaci\'on de 5.0. En todos los casos la gravedad est\'a actuando verticalmente hacia abajo. }
\end{document}
