\documentclass[letterpaper,10pt,onecolumn]{article}
\usepackage[spanish]{babel}
\usepackage[latin1]{inputenc}
\usepackage{amsfonts}
\usepackage{amsthm}
\usepackage{amsmath}
\usepackage{mathrsfs}
\usepackage{empheq}
\usepackage{enumitem}
\usepackage[pdftex]{color,graphicx}
\usepackage{hyperref}
\usepackage{listings}
\usepackage{calligra}
\usepackage{algpseudocode} 
\DeclareMathAlphabet{\mathcalligra}{T1}{calligra}{m}{n}
\DeclareFontShape{T1}{calligra}{m}{n}{<->s*[2.2]callig15}{}
\newcommand{\scripty}[1]{\ensuremath{\mathcalligra{#1}}}
\lstloadlanguages{[5.2]Mathematica}
\setlength{\oddsidemargin}{0cm}
\setlength{\textwidth}{490pt}
\setlength{\textheight}{610pt}
\setlength{\topmargin}{-85pt}
\addtolength{\hoffset}{-0.3cm}
\addtolength{\textheight}{4cm}

\begin{document}
\thispagestyle{empty}
\begin{center}

\includegraphics[width=490pt]{figs/header.png}\\[0.5cm]

\textsc{\LARGE Parcial 3 - F\'isica I (FISI-1018) - Abril 28, 2015}\\[0.5cm]


\end{center}

\begin{enumerate}

\item (20 puntos)
Un peque\~no cubo de masa $m$ se desliza sobre un camino circular de
radio $R$ cortado de un bloque m\'as grande de masa $M$. $M$
inicialmente est\'a en reposo sobre la mesa y todo el movimiento se
hace sin fricci\'on. Encuentre la velocidad $v$ del cubito cuando deja
el bloque.

\item (20 puntos)
Una bola peque\~na de masa $m$ se ubica sobre una bola m\'as
  grande de masa $M$. Las dos bolas se dejan caer desde una altura
  $h$. Calcule a que altura llega la bolita peque\~na despu\'es de la
  colisi\'on. Considere que todos los choques son el\'asticos y que
  $m$ es mucho menor que $M$.

\item (20 puntos)
Los carros $B$ y $C$ se encuentran en reposo y sin los frenos
  puestos. Por detr\'as llega el carro $A$ a alta velocidad empujando
  al carro $B$, y luego el carro $B$ empuja al carro $C$. Si todas las
  colisiones son completamente inel\'asticas y todos los carros tienen
  la misma masa, calcule la fracci\'on de la energ\'ia cin\'etica
  inicial del carro $A$ que se perdi\'o en todos estos choques.  

\item (20 puntos)
Un cilindro macizo de masa $M$ y radio $b$ gira sin
  deslizamiento sobre un plano inclinado un \'angulo
  $\beta$. Encuentre la acelerac\'i\'on lineal con la que baja el
  cilindro. 

\item (20 puntos) Un hombre de masa $m$ va sobre sobre un carro que da vueltas
  sobre un riel circular de radio $R$ a velocidad $v$. Su centro de
  masa se encuentra a una altura $L$ del carro y sus pies est\'an
  separados una distancia $d$. El hombre est\'a mirando en la
  direcci\'on de movimiento. Calcule el peso que reposa sobre cada uno
  de sus pies.

\item (20 puntos) Una barra de longitud $l$ y masa $m$ que se
  encuentre inicialmente en posici\'on vertical empieza a caer sobre
  una superficie sin fricci\'on. Encuentre la velocidad del centro de
  masa cuando el centro ha ca\'ido una distancia $y$ y la barra forma
  un \'angulo $\theta$ con la horizontal.
\end{enumerate}

\begin{figure}[!h]
\begin{center}
\includegraphics[scale=0.25]{figs/parcial3_1.png}
\includegraphics[scale=0.25]{figs/parcial3_3.png}
\includegraphics[scale=0.20]{figs/parcial3_4.png}
\includegraphics[scale=0.20]{figs/parcial3_5.png}
\includegraphics[scale=0.25]{figs/parcial3_6.png}
\includegraphics[scale=0.25]{figs/parcial3_7.png}
\caption{Figuras para cada uno de los ejercicios.}
\end{center}
\end{figure}
{\small {\bf NOTA}: Todas las respuestas deben tener una justificaci\'on
f\'isica y matem\'atica adecuada. 100 puntos corresponden a una nota
de 5.0.}
\end{document}
