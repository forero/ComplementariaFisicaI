\documentclass[letterpaper,10pt,onecolumn]{article}
\usepackage[spanish]{babel}
\usepackage[latin1]{inputenc}
\usepackage{amsfonts}
\usepackage{amsthm}
\usepackage{amsmath}
\usepackage{mathrsfs}
\usepackage{empheq}
\usepackage{enumitem}
\usepackage[pdftex]{color,graphicx}
\usepackage{hyperref}
\usepackage{listings}
\usepackage{calligra}
\usepackage{algpseudocode} 
\DeclareMathAlphabet{\mathcalligra}{T1}{calligra}{m}{n}
\DeclareFontShape{T1}{calligra}{m}{n}{<->s*[2.2]callig15}{}
\newcommand{\scripty}[1]{\ensuremath{\mathcalligra{#1}}}
\lstloadlanguages{[5.2]Mathematica}
\setlength{\oddsidemargin}{0cm}
\setlength{\textwidth}{490pt}
\setlength{\topmargin}{-40pt}
\addtolength{\hoffset}{-0.3cm}
\addtolength{\textheight}{4cm}

\begin{document}
\begin{center}

\includegraphics[width=490pt]{header.png}\\[0.5cm]

\textsc{\LARGE Taller 4 - F\'isica I (FISI-1018) - 2015-10}\\[0.5cm]

\textsc{\Large{Profesor: Jaime Forero}} \\[0.5cm]

\noindent\textsc{Ejercicios correspondiente a la clase complementaria
  de la semana del 16 de Febrero del 2015.}\\[0.5cm]
\end{center}

\noindent\textsc{Nota:} Los primeros cuatro ejercicios deben ser
entregados {\bf al comienzo} de la clase complementaria.  Los tres {\bf
  ejercicios recomendados}  son para trabajo individual o consulta en
la Cl\'inica de Problemas. La numeraci\'on hace referencia al texto
gu\'ia: \textit{F\'isica Universitaria Volumen  1 (Sears-Semansky)},
decimotercera edici\'on, Pearson. 

{\bf Son pocos ejercicios para que puedan utilizar la complementaria
  para hacer un repaso pre-parcial.}  

\begin{enumerate}
\item Ejercicio 4.1 Angulo entre dos fuerzas.
\item Ejercicio 4.8  Viajando en un elevador.
\item Ejercicio 4.19 Sandia en la superficie de Io.
\item Ejercicio 4.27 Das cajas A y B. 
\end{enumerate}

{\bf Ejercicios recomendados}\\
\begin{enumerate}
\item Problema 4.39 Salto vertical sin carrera.
\item Problema 4.51 Salto al suelo.
\item Problema 4.54 Dos bloques conectados por una cadena.
\end{enumerate}


\end{document}
