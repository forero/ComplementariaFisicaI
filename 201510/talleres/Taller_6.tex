\documentclass[letterpaper,10pt,onecolumn]{article}
\usepackage[spanish]{babel}
\usepackage[latin1]{inputenc}
\usepackage{amsfonts}
\usepackage{amsthm}
\usepackage{amsmath}
\usepackage{mathrsfs}
\usepackage{empheq}
\usepackage{enumitem}
\usepackage[pdftex]{color,graphicx}
\usepackage{hyperref}
\usepackage{listings}
\usepackage{calligra}
\usepackage{algpseudocode} 
\DeclareMathAlphabet{\mathcalligra}{T1}{calligra}{m}{n}
\DeclareFontShape{T1}{calligra}{m}{n}{<->s*[2.2]callig15}{}
\newcommand{\scripty}[1]{\ensuremath{\mathcalligra{#1}}}
\lstloadlanguages{[5.2]Mathematica}
\setlength{\oddsidemargin}{0cm}
\setlength{\textwidth}{490pt}
\setlength{\topmargin}{-40pt}
\addtolength{\hoffset}{-0.3cm}
\addtolength{\textheight}{4cm}

\begin{document}
\begin{center}

\includegraphics[width=490pt]{figs/header.png}\\[0.5cm]

\textsc{\LARGE Taller 6 - F\'isica I (FISI-1018) - 2015-10}\\[0.5cm]

\textsc{\Large{Profesor: Jaime Forero}} \\[0.5cm]

\noindent\textsc{Ejercicios correspondiente a la clase complementaria
  de la semana del 2 de Marzo del 2015.}\\[0.5cm]
\end{center}

\noindent\textsc{Nota:} Los primeros tres ejercicios deben ser
entregados {\bf al comienzo} de la clase complementaria.  Los siguientes cinco ejercicios son para trabajo en clase.  
  Los cinco {\bf ejercicios recomendados}  son para trabajo
  individual o consulta en la Cl\'inica de Problemas.  

La numeraci\'on hace referencia al texto
gu\'ia: \textit{F\'isica Universitaria Volumen  1 (Sears-Semansky)},
decimotercera edici\'on, Pearson. 

\begin{enumerate}
\item Ejercicio 5.7. Tensiones sobre una cuerda.
\item Ejercicio 5.15 M\'aquina de Atwood.
\item Ejercicio 5.25 Posici\'on de Trendelburg.
\item Ejercicio 5.10 Pesos y cuerdas inclinadas.
\item Ejercicio 5.14 Trineos.
\item Ejercicio 5.33 Distancia de frenado.
\item Ejercicio 5.42 Din\'amica del movimiento circular.
\item Ejercicio 5.46 Columpio gigante.
\end{enumerate}

{\bf Ejercicios recomendados}\\
\begin{enumerate}
\item Problema 5.56 Arque\'ologo explorador.
\item Problema 5.68 Masas en un plano inclinado.
\item Problema 5.89 Dos bloques conectados por una cuerda ligera.
\item Problema 5.94 Tres bloques.
\item Problema 5.125 Cu\~na m\'ovil.
\end{enumerate}


\end{document}
