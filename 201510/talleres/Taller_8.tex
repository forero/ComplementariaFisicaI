\documentclass[letterpaper,10pt,onecolumn]{article}
\usepackage[spanish]{babel}
\usepackage[latin1]{inputenc}
\usepackage{amsfonts}
\usepackage{amsthm}
\usepackage{amsmath}
\usepackage{mathrsfs}
\usepackage{empheq}
\usepackage{enumitem}
\usepackage[pdftex]{color,graphicx}
\usepackage{hyperref}
\usepackage{listings}
\usepackage{calligra}
\usepackage{algpseudocode} 
\DeclareMathAlphabet{\mathcalligra}{T1}{calligra}{m}{n}
\DeclareFontShape{T1}{calligra}{m}{n}{<->s*[2.2]callig15}{}
\newcommand{\scripty}[1]{\ensuremath{\mathcalligra{#1}}}
\lstloadlanguages{[5.2]Mathematica}
\setlength{\oddsidemargin}{0cm}
\setlength{\textwidth}{490pt}
\setlength{\topmargin}{-40pt}
\addtolength{\hoffset}{-0.3cm}
\addtolength{\textheight}{4cm}

\begin{document}
\begin{center}

\includegraphics[width=490pt]{figs/header.png}\\[0.5cm]

\textsc{\LARGE Taller 8 - F\'isica I (FISI-1018) - 2015-10}\\[0.5cm]

\textsc{\Large{Profesor: Jaime Forero}} \\[0.5cm]

\noindent\textsc{Ejercicios correspondiente a la clase complementaria
  de la semana del 23 de Marzo del 2015.}\\[0.5cm]
\end{center}

\noindent\textsc{Nota:} Los primeros tres ejercicios deben ser
entregados {\bf al comienzo} de la clase complementaria.  Los cinco ejercicios son que siguen para trabajo en clase.   

La numeraci\'on hace referencia al texto
gu\'ia: \textit{F\'isica Universitaria Volumen  1 (Sears-Semansky)},
decimotercera edici\'on, Pearson. 

\begin{enumerate}
\item Ejercicio 7.38 Canica que se mueve sobre el eje x.
\item Problema 7.42 Bloque + resorte + plano inclinado.
\item Ejercicio 8.25 Cazador sobre una superficie de hielo.
\item Problema 7.46 Recorrido por un rizo vertical.
\item Problema 7.56 Cohete sobre una rampa.
\item Problema 7.63 Esquiadora que pierde contacto.
\item Problema 8.31 Choque de asteroides. 
\item Problema 8.48 Dos canicas chocando.
\end{enumerate}


\end{document}
