\documentclass[letterpaper,10pt,onecolumn]{article}
\usepackage[spanish]{babel}
\usepackage[latin1]{inputenc}
\usepackage{amsfonts}
\usepackage{amsthm}
\usepackage{amsmath}
\usepackage{mathrsfs}
\usepackage{empheq}
\usepackage{enumitem}
\usepackage[pdftex]{color,graphicx}
\usepackage{hyperref}
\usepackage{listings}
\usepackage{calligra}
\usepackage{algpseudocode} 
\DeclareMathAlphabet{\mathcalligra}{T1}{calligra}{m}{n}
\DeclareFontShape{T1}{calligra}{m}{n}{<->s*[2.2]callig15}{}
\newcommand{\scripty}[1]{\ensuremath{\mathcalligra{#1}}}
\lstloadlanguages{[5.2]Mathematica}
\setlength{\oddsidemargin}{0cm}
\setlength{\textwidth}{490pt}
\setlength{\topmargin}{-40pt}
\addtolength{\hoffset}{-0.3cm}
\addtolength{\textheight}{4cm}

\begin{document}
\begin{center}

\includegraphics[width=490pt]{figs/header.png}\\[0.5cm]

\textsc{\LARGE Taller 5 - F\'isica I (FISI-1018) - 2015-10}\\[0.5cm]

\textsc{\Large{Profesor: Jaime Forero}} \\[0.5cm]

\noindent\textsc{Ejercicios correspondiente a la clase complementaria
  de la semana del 23 de Febrero del 2015.}\\[0.5cm]
\end{center}

\noindent\textsc{Nota:} Los primeros tres ejercicios deben ser
entregados {\bf al comienzo} de la clase complementaria.  Los siguientes cinco ejercicios son para trabajo en clase.  Para esta semana no hay ejercicios recomendados. 

La numeraci\'on hace referencia al texto
gu\'ia: \textit{F\'isica Universitaria Volumen  1 (Sears-Semansky)},
decimotercera edici\'on, Pearson. 

\begin{enumerate}
\item Ejercicio 4.23 Cajas A y B en contacto.
\item Ejercicio 4.26 Atleta que lanza una pelota.
\item Ejercicio 4.29 Pelota dentro de un tren.
\item Problema 4.35 Dos caballos.
\item Problema 4.41 Biomec\'anica humana. Lanzamiento de b\'eisbol.
\item Problema 4.44 Astronauta.
\item Problema 4.46 Nave espacial que desciende.
\item Problema 4.49 Din\'amica de insectos.
\end{enumerate}


\end{document}
