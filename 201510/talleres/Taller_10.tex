\documentclass[letterpaper,10pt,onecolumn]{article}
\usepackage[spanish]{babel}
\usepackage[latin1]{inputenc}
\usepackage{amsfonts}
\usepackage{amsthm}
\usepackage{amsmath}
\usepackage{mathrsfs}
\usepackage{empheq}
\usepackage{enumitem}
\usepackage[pdftex]{color,graphicx}
\usepackage{hyperref}
\usepackage{listings}
\usepackage{calligra}
\usepackage{algpseudocode} 
\DeclareMathAlphabet{\mathcalligra}{T1}{calligra}{m}{n}
\DeclareFontShape{T1}{calligra}{m}{n}{<->s*[2.2]callig15}{}
\newcommand{\scripty}[1]{\ensuremath{\mathcalligra{#1}}}
\lstloadlanguages{[5.2]Mathematica}
\setlength{\oddsidemargin}{0cm}
\setlength{\textwidth}{490pt}
\setlength{\topmargin}{-40pt}
\addtolength{\hoffset}{-0.3cm}
\addtolength{\textheight}{4cm}

\begin{document}
\begin{center}

\includegraphics[width=490pt]{figs/header.png}\\[0.5cm]

\textsc{\LARGE Taller 10 - F\'isica I (FISI-1018) - 2015-10}\\[0.5cm]

\textsc{\Large{Profesor: Jaime Forero}} \\[0.5cm]

\noindent\textsc{Ejercicios correspondiente a la clase complementaria
  de la semana del 13 de Abril del 2015.}\\[0.5cm]
\end{center}

\noindent\textsc{Nota:} Los primeros tres ejercicios deben ser
entregados {\bf al comienzo} de la clase complementaria.  Los cinco ejercicios son que siguen para trabajo en clase.

La numeraci\'on hace referencia al texto
gu\'ia: \textit{F\'isica Universitaria Volumen  1 (Sears-Semansky)},
decimotercera edici\'on, Pearson. 

\begin{enumerate}
\item Ejercicio 9.2 H\'elice de un avi\'on.
\item Ejercicio 9.10 Ventilador que se apaga.
\item Ejercicio 9.18 Contrapeso de un elevador antiguo.
\item Ejercicio 9.35 Momento de inercia de una rueda.
\item Ejercicio 9.41 Energ\'ia desde la Luna.
\item Ejercicio 9.65 Pelota y disco.
\item Ejercicio 9.84. Dos masas y una polea.
\item Ejercicio 9.86 Autob\'us en Zurich.
\end{enumerate}


\end{document}
