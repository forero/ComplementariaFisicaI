\documentclass[letterpaper,10pt,onecolumn]{article}
\usepackage[spanish]{babel}
\usepackage[latin1]{inputenc}
\usepackage{amsfonts}
\usepackage{amsthm}
\usepackage{amsmath}
\usepackage{mathrsfs}
\usepackage{empheq}
\usepackage{enumitem}
\usepackage[pdftex]{color,graphicx}
\usepackage{hyperref}
\usepackage{listings}
\usepackage{calligra}
\usepackage{algpseudocode} 
\DeclareMathAlphabet{\mathcalligra}{T1}{calligra}{m}{n}
\DeclareFontShape{T1}{calligra}{m}{n}{<->s*[2.2]callig15}{}
\newcommand{\scripty}[1]{\ensuremath{\mathcalligra{#1}}}
\lstloadlanguages{[5.2]Mathematica}
\setlength{\oddsidemargin}{0cm}
\setlength{\textwidth}{490pt}
\setlength{\topmargin}{-40pt}
\addtolength{\hoffset}{-0.3cm}
\addtolength{\textheight}{4cm}

\begin{document}
\begin{center}

\includegraphics[width=490pt]{figs/header.png}\\[0.5cm]

\textsc{\LARGE Taller 7 - F\'isica I (FISI-1018) - 2015-10}\\[0.5cm]

\textsc{\Large{Profesor: Jaime Forero}} \\[0.5cm]

\noindent\textsc{Ejercicios correspondiente a la clase complementaria
  de la semana del 16 de Marzo del 2015.}\\[0.5cm]
\end{center}

\noindent\textsc{Nota:} Los primeros tres ejercicios deben ser
entregados {\bf al comienzo} de la clase complementaria.  Los siguientes 
inco ejercicios son para trabajo en clase.  
  Los cinco {\bf ejercicios recomendados}  son para trabajo
  individual o consulta en la Cl\'inica de Problemas.  

La numeraci\'on hace referencia al texto
gu\'ia: \textit{F\'isica Universitaria Volumen  1 (Sears-Semansky)},
decimotercera edici\'on, Pearson. 

\begin{enumerate}
\item Ejercicio 6.13 Energ\'ia Animal.
\item Ejercicio 6.34 Ni\~na aplicando fuerza a un trineo.
\item Ejercicio 6.44 La mitad de un resorte.
\item Ejercicio 7.2 �A que altura podemos saltar?
\item Ejercicio 7.7 Energ\'ia de humanos contra energ\'ia de insectos. 
\item Ejercicio 7.9 Piedra en un taz\'on.
\item Ejercicio 7.12 Tarz\'an y Jane.
\item Ejercicio 7.14 Resorte ideal de masa despreciable.
\end{enumerate}

{\bf Ejercicios recomendados}
\begin{enumerate}
\item Problema 6.75 Bloque con cuerda sobre una mesa.
\item Problema 6.78 Hombre y bicicleta.
\item Problema 6.85 Bloque sobre resorte.
\item Problema 6.87 Calculando un coeficiente de fricci\'on.
\item Problema 6.88 Arco y flecha.
\end{enumerate}


\end{document}
