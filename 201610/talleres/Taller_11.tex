\documentclass[letterpaper,10pt,onecolumn]{article}
\usepackage[spanish]{babel}
\usepackage[latin1]{inputenc}
\usepackage{amsfonts}
\usepackage{amsthm}
\usepackage{amsmath}
\usepackage{mathrsfs}
\usepackage{empheq}
\usepackage{enumitem}
\usepackage[pdftex]{color,graphicx}
\usepackage{hyperref}
\usepackage{listings}
\usepackage{calligra}
\usepackage{algpseudocode} 
\DeclareMathAlphabet{\mathcalligra}{T1}{calligra}{m}{n}
\DeclareFontShape{T1}{calligra}{m}{n}{<->s*[2.2]callig15}{}
\newcommand{\scripty}[1]{\ensuremath{\mathcalligra{#1}}}
\lstloadlanguages{[5.2]Mathematica}
\setlength{\oddsidemargin}{0cm}
\setlength{\textwidth}{490pt}
\setlength{\topmargin}{-40pt}
\addtolength{\hoffset}{-0.3cm}
\addtolength{\textheight}{4cm}

\begin{document}
\begin{center}

\includegraphics[width=490pt]{header.png}\\[0.5cm]

\textsc{\LARGE Taller 11 - F\'isica I (FISI-1018) - 2016-10}\\[0.5cm]

\textsc{\Large{Profesor: Jaime Forero}} \\[0.5cm]

\noindent\textsc{Ejercicios correspondiente a la clase complementaria
  de la semana del 11 de Abril del 2016.}\\[0.5cm]
\end{center}

\noindent\textsc{Nota:} Los primeros tres ejercicios deben ser
entregados {\bf al comienzo} de la clase complementaria.  Los cinco ejercicios son que siguen para trabajo en clase.

La numeraci\'on hace referencia al texto
gu\'ia: \textit{F\'isica Universitaria Volumen  1 (Sears-Semansky)},
decimotercera edici\'on, Pearson. 

\begin{enumerate}
\item Ejercicio 10.3 Torque sobre una placa met\'alica.
\item Ejercicio 10.17 Caja sobre superficie horizontal.
\item Ejercicio 10.20 Una especie de yo-yo.
\item Ejercicio 10.42 Bloque sobre una mesa con un agujero.
\item Ejercicio 10.48 Choque de asteroide.
\item Problema 10.67 M\'aquina de Atwood.
\item Problema 10.76 Cascaron esf\'erico por una pista.
\item Problema 10.99 Centro de percusi\'on.
\end{enumerate}


\end{document}
