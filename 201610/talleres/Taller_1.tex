\documentclass[letterpaper,10pt,onecolumn]{article}
\usepackage[spanish]{babel}
\usepackage[latin1]{inputenc}
\usepackage{amsfonts}
\usepackage{amsthm}
\usepackage{amsmath}
\usepackage{mathrsfs}
\usepackage{empheq}
\usepackage{enumitem}
\usepackage[pdftex]{color,graphicx}
\usepackage{hyperref}
\usepackage{listings}
\usepackage{calligra}
\usepackage{algpseudocode} 
\DeclareMathAlphabet{\mathcalligra}{T1}{calligra}{m}{n}
\DeclareFontShape{T1}{calligra}{m}{n}{<->s*[2.2]callig15}{}
\newcommand{\scripty}[1]{\ensuremath{\mathcalligra{#1}}}
\lstloadlanguages{[5.2]Mathematica}
\setlength{\oddsidemargin}{0cm}
\setlength{\textwidth}{490pt}
\setlength{\topmargin}{-40pt}
\addtolength{\hoffset}{-0.3cm}
\addtolength{\textheight}{4cm}

\begin{document}
\begin{center}

\includegraphics[width=490pt]{header.png}\\[0.5cm]

\textsc{\LARGE Taller 1 - F\'isica I (FISI-1018) - 2016-10}\\[0.5cm]

\textsc{\Large{Profesor: Jaime Forero}} \\[0.5cm]

\noindent\textsc{Ejercicios correspondiente a la clase complementaria
  de la semana del 25 de Enero del 2016.}\\[0.5cm]
\end{center}

\noindent\textsc{Nota:} 
Los primeros cuatro ejercicios deben ser
entregados {\bf al comienzo} de la clase complementaria. Los \'ultimos
cuatro deben ser trabajados {\bf durante} la complementaria. 

La numeraci\'on
hace referencia al texto gu\'ia: \textit{F\'isica Universitaria Volumen
  1 (Sears-Semansky)}, decimotercera edici\'on, Pearson.

\begin{enumerate}
\item Ejercicio 1.4 (Densidad del oro).
\item Ejercicio 1.11 (Masa cr\'itica de Neptunio).
\item Problema 1.57 (Respiraci\'on de Ox\'igeno).
\item Problema 1.61 (\'Atomos en el cuerpo).
\item Problema 1.64 (Estrellas en el Universo).
\item El r\'ecord de 100 metros planos femeninos los tiene Florence
  Griffith Joyner con una marca de 10.49 segundos, lo que implica que
  recorri� esa distancia con una velocidad media de 9.532 metros por
  segundo. �Cu�l es el valor de la misma velocidad media en kil�metros
  por hora? Atenci\'on a las cifras significativas.
\item La galaxia de Andr\'omeda, ubicada a aproximadamente a una
  distancia de 2 millones de a\~nos luz de la V\'ia L\'actea se acerca
  a nosotros a unos 100 kil\'ometros por segundo. Suponiendo que la
  colisi\'on va a suceder a una velocidad constante (los mismos 100
  km/s que mencion\'abamos antes) �Cu�nto tiempo (en millones de a�os)
  va a pasar para que
  Andr�meda llegue y toque a la V�a L�ctea?\footnote{Les recomiendo
    este video para ver lo que creemos va a pasar con esta colisi\'on de galaxias: \url{https://www.youtube.com/watch?v=qnYCpQyRp-4}}
\item En un libro de astrof�sica me encontr� con la siguiente
  expresi�n
\begin{displaymath}
\alpha  = \sqrt{2GM/r},
\end{displaymath}
donde $G$ es la constante universal de gravitaci\'on, $M$ es la masa
de un objeto y $r$ es su radio. �Qu\'e dimensiones tiene $\alpha$?
\end{enumerate}

\end{document}
